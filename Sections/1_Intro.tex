\section{Introduction}

Throughout history, humanity has focused to become more and more productive. From the moment we as cave-men built our first fires and developed our first tools, we have strived to create bigger and better things. In just roughly 12,000 years (\cite{HumanHistory}), we have walked the moon, built structures like the pyramids and the Burj Khalifa, developed computers that can think for themselves and drive cars, and visited other planets! None of this, however, was possible without the constant persistence to improve and move humanity forward. This focus has also introduced automation several times throughout history. Therefore, we should look at this past history as well as current trends to explore 3 stages of automation in our near future and the many ways society and humanity can deal with the dramatic change in labor and economy. Initially, humanity will find jobs for those who are displaced via retraining. As time goes on, jobs will continue to be removed and a universal basic income will be placed to allow for all to survive. The last jobs will be focused in creativity and personability while working alongside AI and robotics to push humanity to new heights. Let's take a trip down a potential future.

\subsection{Learning from History}

It is generally accepted that there have been 3 major industrial revolutions throughout history, although humans have been attempting to mechanate things since the time of Greek empire (\cite{AutomationPeriods}). The first industrial revolution started in 1765, and included the mechanization of agricultural tools, like the horse-drawn cradle. A century later, humanity industrialized once again, this time grasping control over energy in the form of electricity, oil, and gas. This lit up cities, enabled the development of cars and planes, and most importantly created new machines to increase produced goods. The age of the production line was now here. Fast forward another 100 years, and the age of computers had arrived. With this came basic robotics, PLCs (programable logic computers), and of course the internet. However, while we enjoy the fruits of our past revolutions, the next (and potentially final) one is looming around the corner: the Robotics and Artificial Intelligence Industrial (AI) Revolution. And it is coming much faster than one may think.

\subsection{Why is this important?}

To prepare for the societal changes this revolution will bring, we can take a look at our history to preemptively plan for our future. While we have been able to live "in the moment" without any plan in the past, advanced robotics and artificial intelligence allows for a sort of 'end game,' where we have the capability to no longer force people to do work (See \hyperlink{sec:long}{Long-Term Section})!

While our potential in the future may seem exciting, it is important to look at the alternative as well. While some may say that it has just worked out in the past, we can't rely on such a mentality this time around. Without planning, we risk an extreme wealth and power gap, never before seen in history. The few owners, developers, and maintainers of the advanced and artificially intelligent robots will profit at an unprecedented rate if no regulation, taxation, or other plan is created. In the past, we've been able to use government regulation to control the outliers in capitalism, like John D. Rockefeller (oil), John Jacob Astor (real estate), JP Morgan (banking) and Andrew Carnegie (steel) (\cite{CapitalistAmerica}). However, the potential billionaires and trillionaires in the future can have so much power in society that we may have no choice but to bend to their demands. A quick look at history depicts a warning; JP Morgan, Rockefeller, and Andrew Carnegie infamously had strong control in late 19th century and early 20th century American politics, until \textbf{PRESIDENT} famously introduced many governmental reforms to attempt to prevent such an issue. % TODO: add president + citation
% See this paper for source: https://books.google.com/books?hl=en&lr=&id=MRg5crpAOBIC&oi=fnd&pg=PR2&ots=oKVvVj8ksY&sig=uWeQNMdXNjOGxNvRC8p4S9pvMFo#v=onepage&q&f=false

While government regulation may be a suitable solution, I truly believe societal change is the correct answer to this threat. By implementing a fundamental change in the way we develop, share, and profit from software (and maybe even hardware), we can democratize innovation, and prevent the possibility of capitalistic control as seen in the past. The open-source movement is a great current-day example (\hyperref[subsec:open-source]{see Open-Source Movement under the Mid-Term Section}) - a movement I am particularly very passionate and involved in. As we move forward in the future, we will see the % TODO: Finish this section!

\subsection{Robotics and AI is coming, and it's coming fast!}
There is no question that robotics is a booming industry.  The growth of robotics has been compared to the Cambrian Explosion, where life on Earth experienced a very short but rapid growth and diversification (\cite{Cambrian}). Similarly, the field of robotics - and by extension Artificial Intelligence (AI) - has benefited to the growing complexity of complicated algorithms as hardware (like microelectronics and single-board computers) becomes cheaper, more accessible, and more powerful. % TODO: Add figure of Cambrian curve here

This will result in more automation, reducing costs of goods and allowing for humanity to focus on what they do best: creativity. However, this also will remove many of the jobs people currently hold. Communities will therefore seek to adapt to the inevitable automation (\cite{UnionsAutomation}), as people are forced to adjust to the new reality.