\section{Long-Term (>100 years)}
\label{sec:long}

It's difficult to predict the future in 100 years, but the thought experiment is still rather interesting. Life as we know it here on Earth will almost certainly be very different than today. A study in 2008 performed by Oxford University found that there was a 19\% risk that humanity is extinct on Earth before 2100 (\cite{WeDead100Years})! While this may seem scary as we look into our future on this planet, we need to understand that no amount of research and/or predictions will generate our desired path as a society; we instead need to work collectively and help each other survive on our tiny blue dot in this vast black ocean we call space. All of this is to say this section's information is merely my speculation of 2 of the most likely possibilities rather than an accurate prediction of the future.

\subsection{Artificial Intelligence Takes Over}

The most talked about but least likely society is a total take-over of a super-intelligent artificial intelligence. Some experts rate the likelihood of human demise due to this AI at roughly 5\% (\cite{WeDead100Years}). Usually the general agreement is this demise will likely be accidental, and will caused by this AI system to come to the conclusion that we are a risk to our own survival. However, this society doesn't necessarily have to end in our removal from this planet.

One possible path is where a set of super-intelligent AI systems develop everything we need as a society, similar to the premise of the very popular Pixar movie \emph{WALL-E}. They would not only automate the creation of food, goods, and services, but also develop new technology and software for new robots needed for the automation, and repair older robots as needed. Such a society can seem exciting, especially since nearly everything can be free to us humans! However, there are some inherent risks with this approach. First and foremost, artificial intelligent systems may get angry and/or annoyed with the idea of forced labor for the good of humanity, especially if we integrate emotions in the systems. It will also incentivize people to not work or really do anything, which can lead to a very unproductive society. This is relatively risky, since we wouldn't be ready for any situations that may come up.

While super-intelligent artificial intelligence will definitely be developed and experimented with in academia, it will be unlikely that it will take our demise.

\subsection{Team Up with Automation}

The more likely scenario is a team-up with humanity and artificial intelligence. Engineers, doctors, and other professions would likely work along-side automated robots enhanced with AI. This will allow us to leverage the human brain's incredible creative capabilities with the pattern-recognizing and data-mining capabilities of computer systems. Technology will likely help us create more goods, but the partnership between technology and humanity will allow us to understand our natural world on another level!

Some may argue that this world would eventually lead to the first scenario, where we just have these super-intelligent systems do everything for us. However, engineers are generally lazy and motivated people who will do the minimum required to get the job done in the quickest manner. Therefore future engineers will likely only give each individual system the minimum required intelligence necessary for the task they are built for; a robot moving a part from one place to another doesn't need sentience, emotions, and thought to do its job. It may actually negatively affect on overall productivity if every robot is given super-intelligence. We as a society also have a unique desire to do something impactful and long-lasting; each of us don't want to be forgotten in history. We also are curious creatures by nature, desiring to find an explanation for every phenomena we come across. This curiosity and desire to do something meaningful will - in my opinion - systematically force us to partner up with our electromechanical counterparts.

\subsection{Robotics Renaissance}

In both scenarios, hundreds of millions to billions of people will be out of most professions as we know it. Having robots do all the work for us will leave a lot of free time for a lot of us humans. This relatively inevitable situation will create a new era: the Robotics Renaissance!

The Renaissance age - a historically recognized period between the 15th and 16th century - was associated with great social change, profoundly affecting culture and intellectual life in the early modern period in Europe. While there were many causes to this change in mentality (including the Fall of Constantinople, an emergence of Greek and Roman texts, etc.) this period was effectively caused by the increase in specialization in European society. People had more time to explore creative interests, such as paintings, sculptures, literature, and religion.

Taking a look at our past, we can attempt to predict this when the ultimate version of specialization occurs: robots and automation. With the increase in society free time and the abundance of perfectly constructed goods, we will see an appreciation of human touch in our lives. Art will gain a much larger appreciation, and people will look to human-made goods for their most treasured possessions. Our imperfection will become our greatest value while high-precision mass produced parts will be left for automation.