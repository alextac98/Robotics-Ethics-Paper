\section{Conclusion}
\label{sec:conclusion}

Our future may appear both bright and frightening as we charge forward with developing advanced technologies and dealing with the gremlins of the past. In the next few decades, we as a society will need to find a solution to pollution, develop some of the most advanced software and hardware systems, explore the depths of space, and protect our people from the threat of major disease. However, as we achieve bigger and better things, we will also need to deal with the social issues that it all comes with. One of the most pressing issues will be the diminishing of jobs as automation increases with the creation of more advanced artificial intelligence and robotics systems.

Initially, we will see some of the most grueling and tough jobs automated, including warehouse, factory, and fast-food jobs. This will effect a significant amount of low-income people, who will need to find other jobs. On a brighter side, the increase in automation will also see the reduction of costs and an increase in higher-quality jobs in the form of software engineers, robot technicians, and mechanics, where problem solving is more prominent. In order to ensure everyone in our society has a living wage, retraining programs should be implemented to move displaced workers to more stable and higher quality jobs. We should also consider a staged universal basic income to prepare for a future where less and less jobs are available.

As we approach the middle of the century, automation will start becoming advanced enough where any mundane job will be taken. At the same time, there will be an increase in jobs that require creativity, innovation, and empathy. This will include cooks, doctors and nurses, and engineers. This time period will be the first where we see many people lose jobs and choose not to work. It will also be a period of extreme growth for companies that develop the automation technologies that will become ubiquitous. Therefore, society risks an ever-increasing separation of economic wealth. A proposed solution would be to embrace the open-source movement, and make information and technology as publicly available as possible. This will shift knowledge to the general populous, and put pressure on companies and their owners to innovate or risk competition. It will also allow for technology to be developed faster as we share the knowledge we gain from the experiments we perform. A staged universal basic income could also help society avoid an extreme wealth gap between the upper and middle/lower classes.

While there are many threats humanity will need to avoid in the next century, history has shown our tenacity to survive and thrive, and there is no reason to believe we will be any different. The future looks interesting once automation becomes so advanced we won't need to work. Two different paths will arise for humanity once we reach this intersection. The less likely will be a world where super-intelligent robots will do the work for us. Humanity will essentially be able to live for free while we reap the benefits of our technological development in the past. However, the curious and lazy nature of humans makes this path rather improbably, especially when compared to the more likely scenario. Society will likely develop advanced robotics and artificial intelligence to accompany us in our constant pursuit of innovation and understanding of the natural world. Doctors, nurses, and engineers will utilize these advanced tools to allow for an extremely high quality of living for everyone on the planet, not just first-world countries. Society will also grow to appreciate the imperfectness of humans, a phenomena I like to call the Robotics Renaissance. Human-made goods will be valued extremely highly, including furniture, art, and day-to-day goods.

Many of us are worried about automation taking our own jobs. While this threat is very real, we should take a step to realize the real threat: losing our livelihoods. While the most obvious solution is to avoid automating jobs all together, this is the worst path a society can take, since it stifles innovation and relies on imperfect people to perform hard labor that often can be dangerous to their health. As history has proven time and time again, anything path that will improve overall output will be the path that humanity takes. We as a society should accept the inevitability of automation, and take measures in advance to ensure humanity's prosperity.