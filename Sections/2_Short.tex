\section{Short-Term (< 20 years)}
\label{sec:short}

With the current growth of technology, it will not be surprising if most jobs are affected by automation in some way in the next 20 years (\cite{1/2Jobs20Years}). At the heart of every one of these jobs is a person who relies on the paycheck they receive to stay healthy and happy. While the idea of optimizing our society with automation to make high quality things more affordable and available, it is extremely important to remember those who will suffer from the change. According to a statistical report written by Oxform in 2013, nearly half of jobs may be replaced by some form of automation (\cite{OxfordJobsAreComputerized}). This includes computer programs that automate tasks, physical robots interacting with the physical world, and artificially intelligent systems that decide the best path to take. So what exactly will the world look like by 2040?

\subsection{Outlook on Robotics and Jobs by 2040}

While most of us agree that robots will take over jobs in the future, it is hard to predict exactly what will change in 20 years. The first jobs to be automated are those that either pose a health risk (ex. repetitive injuries, heavy lifting risks, etc.) or are very mundane that doesn't require extreme amounts of dexterity. Most factory and warehouse jobs will fall underneath this category, including material handlers and some general assembly associates. Other minimum wage jobs may also suffer in the next 20 years. David Autor sees most of these jobs reduced and replaced with robotics (\cite{WhyStillSoManyJobs}). The reduction of minimum wage jobs will also hit young people the hardest, since most of these jobs are held by those who are trying to get through college or are just out of high school, as they act as stepping stones to more permanent professions. 

On a more positive note, many jobs will thrive and grow! Software development and engineering positions will obviously benefit in this moderately robotic future. However, several professions may actually indirectly boom with the addition of automation. According to a recent report by the U.S. Bureau of Labor Statistics, healthcare jobs will thrive in the next decade. While initially this may seem unintuitive, artificial intelligence and robotics will work alongside doctors and nurses to better detect and treat diseases and illnesses  (\cite{USBLS_JobsIn10Years}). For example, a mixture of artificial intelligence algorithms in communication with a physician and an MRI (Magnetic Resonance Imaging) compatible robot can take significantly more accurate biopsies of cranial (brain) cancer patients (\cite{Fischer_MRI_Robot}). The current method has a surgeon look at MRI images and blindly guess a location and depth for needle insertions. Technologies like this won't replace jobs, but rather improve accuracy and quality of care. It shows that automation can work alongside people to make a better place and world.

\subsection{Potential Solutions for Job Loss}

While the outlook on minimum wage labor looks bleak, the U.S Bureau of Labor Statistics actually predicts that the number of jobs will increase by roughly 6 million (\cite{USBLS_JobsIn10Years}), even though the market will look very different from what it is now. Since people will be displaced from their regular jobs, society should put in place some plans to avoid social and economic backlash from the change of labor.

\subsubsection{Retraining Programs}
One of the most proposed solutions to the issue of labor displacement caused by automation has to be retraining programs. This highly debated topic argues that either government and/or companies who implement job-removing automation should create retraining programs for displaced workers. This seems like a win-win for all involved, and encourages a more educated and specialized society. However, many issues remained unanswered with current implementations. The first is the argument of desire. Some people argue that those who's jobs are displaced don't want to be retrained in another field, but rather want to stick to their original job (\cite{miller2017beatRobots}). However, most jobs removed during this time period would be minimum wage or close to minimum wage positions usually done by those who have minimal education and few choices to switch to a more desireable career path. Essentially, those who work these types of jobs don't work because they want to but rather because they have to. Introducing them to a more technical field - like low-level software development and mechanical drafting - is a great way to give people who may be struggling a chance at a higher paying carrier. One great example of this is a non-profit organization working to retrain coal miners to website development. Although it is too early to tell, initial indications show that the people who have gone through the program have ended up living a healthier, wealthier, and happier life (\cite{phamRetrainingUnemployed}).

The question of who would pay for these retraining programs is another topic of contention. Some argue that it is up to companies to retrain those who are laid off. A government mandate could force retraining through non-profit organizations that could ensure the quality of classes. However, companies are likely to spend minimal effort and money to properly give their previous employees the training needed in a field that interests each individual. Rather, the government should be responsible to manage unemployment and work to encourage those who are recently displaced. The question of how governments should pay for these efforts lay in the idea of robot taxation. 

\subsubsection{Robot Taxation} 
This innovative idea places additional costs on companies who choose to replace human jobs with automated ones by taxing each robot or system of automation per hour of work. This incentivize automation that replaces human jobs while paying for programs that do the work itself (\cite{abbottShouldRobotsPayTaxes}). Governments can then decide how to best distribute funds to help society with the increase in automation. Proponents also argue that it will better distribute wealth from the richest companies to the lower class in a society. However, the plan has several fundamental flaws. Such a tax encourages companies to avoid automation unless absolutely necessary, which would stifle innovation and potentially leave a country behind technologically. It also poses a difficult question of which robot should be taxed and which should not be taxed, especially when the definition of a robot or an artificially intelligent system is very loose. Finally, taxing robots and other forms of automation could bring voting into question. The argument that robots should be represented in government since they are taxed could be brought up, with similar claims as what the American settlers brought up to their English king back in the late 18th century. Ultimately, the question of robot taxation is a discussion our society will need to have. I believe each country will implement their own version of this law. However, those with the smallest tax (or no tax at all) will ultimately lead the technological development race.

% \subsubsection{Introduction of Universal Basic Income}
