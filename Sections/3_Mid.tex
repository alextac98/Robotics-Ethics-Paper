\section{Mid-Term (20 - 100 years)}
\label{sec:mid}
This in-between period will prove to be a defining period for humanity. If every time period this paper talked about had a theme, this one would probably be "Action or Separation." With ever-increasing globalization for both technology and culture, we will either see an extreme separation of wealth or a great equalizer where nearly everyone has equal opportunity to compete. Experts generally agree that such a difference in wealth will shift power to just a few individuals - both politically and socially - in a way that can't be balanced or checked (\cite{WhatsWrongWithWealthDistribution}). We saw something similar in the late 19th and early 20th century, with the business powerhouses of Rockefeller, Morgan, and Carnegie, where top industry leaders exuded significant political and social power in the United States. So what may jobs look like in 50 years, and how can we prevent an extreme wealth gap?

\subsection{Outlook on Robotics, Jobs, and Society}

As humanity develops more and more advanced automation technologies, it is hard to envision a world where robots don't produce goods faster and better than their human counterparts. However, it's interesting to explore the jobs that will remain. Most jobs held by humans during this time period will likely be either creative or social (\cite{JobsNoAI}). These may include theoretical and/or experimental researchers, engineers, doctors, nurses, teachers, and chefs. Nearly every manual labor requiring occupation will likely be replaced by automation, including but not limited to grocery store clerks, fast food workers, and factory associates. However, automation may move into more technical jobs as well, which may lead to an increase in their use. A research paper in 2014 looked at the effect that robotics and artificial intelligence may have in the health care sector. While jobs requiring emotional sympathy will absolutely be necessary (especially in emotion-based careers such as doctors and nurses), robotics and artificial intelligence will grow the field as a whole. We will likely see automation work along-side human nurses and doctors to improve overall quality of care (\cite{FutureMedicalRobotics}). Similar trends are also expected in other technical fields, such as research scientists and engineers, where automation will reduce mundane tasks and help innovators discover new technology. This explosive growth in tech development, however, may help some grow more powerful than any other human in the past.

It's no question that business leaders who control large corporations have more social and political power in today's capitalist societies. With the introduction of more and more automation, it is important we try to spread this control over as many people as possible to reduce the likelihood of one person having too much control (\cite{WhatsWrongWithWealthDistribution}). If a single person can gain control over a significant portion of the automation market share, they will hold more economic power than any other person in the world: a dangerous proposition for an unelected and relatively unregulated individual. In addition, since automation is at the center of their reign, it will be easier and easier to become more and more powerful, since they can use their own technology to become more and more efficient. While this scenario may not necessarily lead to a bad situation, absolute power has been known to corrupt absolutely. it is therefore up to us as a society to create systematic changes socially, technically, and politically to avoid an extreme outcome in our future.

\subsection{Wealth Distribution Solutions}

So how can we as a global society equalize economic and social opportunities as best as possible? Author Matshona Dhliwayo is attributed to saying "Knowledge is wealth, wisdom is a treasure, understanding is riches, and ignorance is poverty." Therefore, in order to avoid a power gap, we should systematically implement wealth distribution solutions. Some keen-eyed readers may say this sounds like a socialistic society, but there is one major difference: information vs material. Socialism, as defined by Busky, is characterized by the shared ownership of production and enterprises (\cite{WhatIsSocialism}). While many interpretations have existed, many radical implementations of socialism have largely failed, such as most communist regimes. These implementations have focused on shared ownership of goods and organizations through government mandates. A relatively free and competitive market, however, encourages competition and innovation, which pushes humanity forward. While some government regulation may help redistribute material wealth through universal basic incomes, the most powerful way of paving a rich path for humanity is by encouraging the free and open sharing of information and technology.

\subsubsection{The Open-Source Movement} 
\footnote{Bias Disclaimer: I am a big supporter of the open-source movement. I often contribute to open-source projects and develop software and hardware for my personal open-source projects.}
\label{subsec:open-source} 
A potentially different mentality should be used for our future: sharing information rather than ownership. This encourages innovation since inventors can use existing technology to develop new concepts, and forces competition through the requirement of being the best in order to continue. The sharing of information also reduces the likelihood of powerful technology leaders from controlling society. This is exactly what the open-source movement attempts to do: share information, knowledge, technology, and code with others in the world. Once someone creates a useful and impressive project, a community can support the developer by contributing to their codebase, finding and reporting bugs and crashes, and donating to help the developer keep developing. While the software is usually the center, the open-source movement does not specify the type of knowledge or technology to be shared. For example, most consumer and prosumer level 3D printers have open-sourced hardware and software. Many of today's most popular operating systems are also open-source, including Android and Linux. The spirit of the movement focuses on the free sharing of information and technology rather than just code or just mechanical designs. The open-source community also works on a pure meritocracy. It doesn't matter where or who the developer is; if the code to be merged into the code base is a positive improvement in the eyes of the maintainers, it will be added.

So how is this relevant to robotics and jobs? The open-source movement is a systematic method of preventing a large gap in power, where the uber-rich control everything while most people are in poverty and under their control. Best of all, the open-source movement can be grown systematically, not politically. This directly decentralizes and democratizes the growth of a movement, free from any political or socioeconomic pressure. By open-sourcing technology and knowledge, it will help us as a society move forward faster - since we will be essentially sharing notes - while pushing for a competitive market where inventors are encouraged to innovate. 

Critics often have two major complaints of the open-source movement: quality and monetization. Some will argue that it is irresponsible and impossible to use open-source projects and information since it can be unreliable, low-quality, and in some cases misleading (\cite{OpenSourceFail}). They argue that the best software is developed by a well-funded team, and we cannot rely on random engineers with unknown backgrounds to develop great software. This argument misses the point of open-source culture. While there are definitely some underwhelming projects, some of the most used and best software platforms have been developed using the open-source nature. The best projects will attract the best engineers, which will work to ensure the project only accepts the best of contributions. A study published in 2017 argues that the open implementation and process of an open-source project encourages quality participation in its users (\cite{OpenSource_Good}). There is no better example than the most popular and most secure operating system (OS) in the world: Linux. Developed and maintained by Linus Torvalds, Linux was meant to be a replacement to Unix: a paid OS. Even though it started after Microsoft's Windows, it is by far the most popular operating system in the world, helped by Android: another open-source operating system that uses the Linux kernel! Most servers run a version of Linux due to its stable and open nature, especially when compared to Microsoft's counterpart: Windows Server.

The second argument critics have is the monetization of open-source projects, namely how can people make livings giving away knowledge? Many companies have adapted to support the open-source community while still making a lot of money. Microsoft, Google, and Facebook are among the biggest contributors to open-source projects (including Linux, Ubuntu, React, and TensorFlow) and are some of the most profitable companies in the world. So how do these projects support themselves? The easiest ways are usually advertisements, donations, and/or licenses (\cite{OpenSouceMakeMoney}). A great example is a project called ChibiOS: an operating system for microcontrollers. While open-source, the operating system is licensed for free for hobbyists, educators, and projects that use the GPL3 license while requiring a contract with companies who try to commercialize their product that utilizes ChibiOS (\cite{ChibiOS}). Another great way for open-source projects and companies to be self-sufficient is to follow a service model instead of a product model. The main product sold could be support service rather than the product itself. Canonical - makers and maintainers of the most popular Linux distribution Ubuntu - follow this exact model. Instead of selling the operating system to their customers, they offer it for free and make it open source, and sell support for it instead. Big businesses will usually purchase this support so they don't have to worry about fixing issues they are unfamiliar with, while others can get a top of the line operating system for free! Since it is open-source, customers can submit their own changes to make the operating system better for everyone. 

\subsubsection{Universal Basic Income}

While the open-source movement can help prevent an extreme difference in wealth between the upper and lower classes, more should be done to ensure that everyone is able to live comfortably with the wages they have, especially if nearly every job is at risk. When looking at the few options we have to support everyone under a non-communist society, universal basic incomes (UBIs) look like one of the only options we currently have. A Stanford analysis recently uncovered that UBIs, or a similar system, will become more and more necessary, especially as the very common "minimum wage" jobs are lost to automation (\cite{StanfordUBI}). While they may not be very reasonable with today's level of automation, it isn't hard to imagine a time period where most jobs as we know are gone.

Today's opponents of the concept of a Universal Basic Income have a few major arguments: cost and necessity. It is estimated that if every person received \$1,000 per month, it would cost the United States government nearly \$2.4 trillion annually, or 1/8th of the total GDP of the country. This extreme cost will also divert assistance to those who need it most, since social services may suffer (\cite{UBI_WhyNot}). This tremendous cost, however, can be reduced significantly by creating a staged UBI, where total income reduces the UBI received per individual. The family's who need it most would receive more, while the upper-middle class would see a reduction in their UBI. It can also be funded by taxing companies more stringent and creating laws that more effectively prevent legal tax avoidance. UBI opponents also argue that it is not necessary, as jobs will change as the time comes by, and that it would shrink the labor force (\cite{UBI_WhyNot}). However, this argument is based on current facts - not future predictions of the job market. With the ever-growing use of automation, it is a matter of when - not if - jobs will be replaced by a robot. While the details of a universal basic income have certainly not been worked out, we as a society should consider solutions like UBIs to support the working lower class when they inevitably lose their jobs.