\documentclass[12pt,letterpaper,man,natbib]{apa6}

\usepackage[english]{babel}

\usepackage{outlines}

\usepackage[utf8]{inputenc}
\usepackage{amsmath}
\usepackage{graphicx}

\title{\large{The Future of Robotics and Labor Proposal}}
\shorttitle{Future of Robotics and Labor | Tacescu}
\author{Alex Tacescu}
\affiliation{HUA 4900 Robotics, AI, and Ethics}
\date{\today}

\begin{document}
    \maketitle

    \textbf{Introduction}

    There is no question that robotics is a booming industry. In fact, the growth of robotics has been compared to the Cambrian Explosion, where life on Earth experienced a very short but rapid growth and diversification (\cite{Cambrian}). Similarly, the field of robotics - and by extension Artificial Intelligence (AI) - has benefited to the growing complexity of complicated algorithms as hardware (like microelectronics and single-board computers) becomes cheaper, more accessible, and more powerful. This will result in more automation, reducing costs of goods and allowing for humanity to focus on what they do best: creativity. However, this also will remove many of the jobs people currently hold. Communities will therefore seek to adapt to the inevitable automation (\cite{UnionsAutomation}), as people are forced to adjust to the new reality. 

    \textbf{Short Term: Robots Take Over Manufacturing}
    
    With the current growth of technology, it will not be surprising if most jobs as we know it will be different and/or gone within the next 20 or so years (\cite{1/2Jobs20Years}). In fact, a report by Oxford written in 2013 shows that nearly half of all jobs may be replaced by some form of automation, whether that is a computer program or a physical robot interacting with the physical world (\cite{OxfordJobsAreComputerized}). The change will happen too quick for capitalist societies to react. However, the increase in automation will also increase demand for more technical jobs. Currently, startups like BitSource look to retrain existing labor markets to more technical jobs like software developers (\cite{CoalToCode}). Meanwhile, we will see many other opportunities we cannot hope to predict, similar to the manufacturing revolution in the early 20th century and the 18th century(\cite{IndustrialRobotRevolution}).

    \textbf{Mid Term: Robots Take Over Most Non-Creative Jobs} 
    
    A common quote said in the robotics community has been: "the last job will be a roboticist." Therefore, this middle period will likely see most jobs be creative and/or personal (\cite{JobsNoAI}). These may include theoretical and experimental researchers, doctors, teachers, and chefs. Meanwhile, the rest of the population may need to rely on a universal basic income (UBI) to keep capitalism going in the same direction. A Stanford analysis recently discovered that UBIs will become more and more necessary, especially as common "minimum wage" jobs are lost to automation (\cite{StanfordUBI}). Other people, however, believe that retrainng existing markets will be the key to keeping people engaged in society as the labor market shifts with automation.

    \textbf{Long Term: Technology Renaissance}
    
    The state of society when robotics can take over all jobs will take one of two potential paths: a robot-controlled future and a collaborative future. We can imagine that some sort of basic income would be implemented from the previous term, but jobs may look very different. One argument for this future is that humanity will develop AI to the level that it no longer needs to work. Software will write software, develop hardware, and introduce new ideas and technologies (\cite{FutureOfAI}). On the other spectrum, AI will work alongside humans, complementing each others' strengths and weaknesses to achieve more in less time (\cite{AI_Humans_Future}). In either situation, humanity will likely see a Renaissance, where people start valuing human-made goods because the supply of it is significantly less.

    \textbf{Thesis Statement}

    This paper will study past history as well as current trends to explore 3 potential periods of automation in our future and the many ways society can deal with the dramatic change in labor. Initially, humanity will find jobs for those who are displaced via retraining. As time goes on, jobs will continue to be removed and a universal basic income will be placed to allow for all to survive. The last jobs will be focused in creativity and personability while working alongside AI and robotics to push humanity to new heights.

    \pagebreak
    \textbf{Outline}
    \begin{outline}[enumerate]
        \1 Introduction
            \2 Why is this important?
                \3 People are scared of losing jobs!
                \3 Need to ensure wealth gap is minimized
                    \4 Explain why the wealth gap needs to be minimized
                \3 Need to plan how to sustain society with the potential of not having a job
            \2 Speed of Robotics
                \3 Prove Speed of Robotics is increasing with studies
                \3 Compare to the Cambrian explosion (\cite{Cambrian})
        \1 Short-Term (<20 years)
            \2 Outlook on Robotics and Jobs
                \3 Physical and mundane jobs replaced by automation
                \3 Most people not displaced
            \2 Potential Solutions for Job Loss
                \3 Retraining Programs
                \3 Robot Taxation
                \3 Introduction to UBI (Universal Basic Income)
        \1 Mid-Term (20-100 years)
            \2 Outlook on Robotics and Jobs 
                \3 Most non-creative jobs are replaced by automation
                    \4 Creative jobs include chefs, artists, graphic designers, etc.
                \3 Technical jobs not replaced
                    \4 Examples include software developers, engineers etc.
                \3 Problem solving jobs not replaced
                    \4 Examples include IT, mechanics, electricians, builders etc.
            \2 Solutiosn for Job Loss
                \3 UBI becomes more necessary in jobs
                \3 Open Source software can distribute wealth gap!
        \1 Long-Term (>100 years)
            \2 Outlook on Robotics and Jobs
                \3 Two different outcomes forseen:
                \3 Robotics and AI take over all jobs
                    \4 Everything can be free given enough resources
                    \4 People can get a certain amount of money to purchase what they need and want (UBI)
                \3 Robots and AI work alongside humans
                    \4 Explain why more likely
                    \4 Only people who want to work would work!
            \2 Introduction to the Robotics Renaissance
                \3 Name may change by the time paper is written
        \1 Conclusion
            \2 Overview of 3 time periods
            \2 Reiterate thesis statement
    \end{outline}

    \bibliography{ref.bib}

\end{document}