\section{Mid-Term (20 - 100 years)}
\label{sec:mid}
This in-between period will prove to be a defining period for humanity. If every time period this paper talked about had a theme, this one's would be %TODO: Come up with a theme.
With an ever-increasing globalization for both technology and culture, we will either see an extreme separation of wealth or a great equalizer where nearly everyone has equal opportunity to compete. Experts generally agree that such a difference will shift power to just a few individuals, both politically and socially in a way that can't be balanced or checked (\cite{WhatsWrongWithWealthDistribution}).

\subsection{Outlook on Robotics and Jobs}

Most jobs are lost to Robotics

Wealth distribution intro - people can be VERY powerful

Why is wealth distribution bad

How can it happen

What would the world look like with wealth distribution

\subsection{Wealth Distribution Solutions}

So how can we as a global society equalize economic and social opportunities as best as possible? Author Matshona Dhliwayo once said "Knowledge is wealth, wisdom is treasure, understanding is riches, and ignorance is poverty." % TODO: Add citation
Therefore, in order to avoid a power gap, we should systematically implement wealth distribution solutions. Some keen-eyed readers may say this sounds like a socialistic society, but there is one major difference: information vs material. Socialism, as defined by Busky, is characterized by the shared ownership of production and enterprises (\cite{WhatIsSocialism}). While many interpretations have existed, many radical implementations of socialism have largely failed, such as most communist regimes. These implementations have focused on shared ownership of goods and organizations through government mandates. A relatively free and competitive market, however, encourages competition and innovation, which pushes humanity forward. While some government regulation may help redistribute material wealth through universal basic incomes, the most powerful way of paving a rich path for humanity is by encouraging the free and open sharing of information and technology.

\subsubsection{Open-Source Movement} 
\footnote{Bias Disclaimer: I am a big supporter of the open-source movement. I often contribute to open-source projects and develop software and hardware for open projects.}
\label{subsec:open-source} 
A potentially different mentality should be used for our future: sharing information rather than ownership. This encourages innovation since inventors can use existing technology to develop new concepts, and forces competition through the requirement of being the best in order to continue. The sharing of information also reduces the likelihood of powerful technology leaders from controlling society. This is exactly what the open-source movement attempts to do: share information, knowledge, technology, and code with others in the world. Once someone creates a useful and impressive project, a community can support the developer by contributing to their code base, finding and reporting bugs and crashes, and donating to help the developer keep developing. While software is usually the center, the open source movement does not specify the type of knowledge or technology to be shared. For example, most consumer and pro-sumer level 3D printers have open-sourced hardware and software. Many of today's most popular operating systems are also open-source, including Android and Linux. The spirit of the movement focuses on the free sharing of information and technology rather than just code or just mechanical designs. % TODO: Add something about equality in contributors

So how is this relevant to robotics and jobs? The open source movement is a systematic method of preventing a large gap in power, where the uber rich control everything while most people are in poverty and under their control. Best of all, the open source movement can be grown systematically, not politically. This directly decentralizes and democratizes the growth of a movement, free from any political or socioeconomic pressure. By open sourcing technology and knowledge, it will help us as a society move forward faster - since we will be essentially sharing notes - while pushing for a competitive market where inventors are encouraged to innovate. 

Critics often have two major complaints of the open-source movement: quality and monetization. Some will argue that it is irresponsible and impossible to use open-source projects and information, since it can be unreliable, low-quality, and in some cases misleading (\cite{OpenSourceFail}). They argue that the best software is developed by a well-funded team, and we cannot rely on random engineers with unknown backgrounds to develop great software. This argument misses the point of the open-source culture. While there are definitely some underwhelming projects, some of the most used and best software platforms have been developed using the open-source nature. The best projects will attract the best engineers, which will work to ensure the project only accepts the best of contributions. A study published in 2017 argues that the open implementation and process of an open-source project encourages quality participation in its users (\cite{OpenSource_Good}). There is no better example than the most popular and most secure operating system (OS) in the world: Linux. Developed and maintained by Linus Torvalds, Linux was meant to be a replacement to Unix: a paid OS. Even though it started after Microsoft's Windows, it is by far the most popular operating system in the world, helped by Android: another open-source operating system that uses the Linux kernel! Most servers run a version of Linux due to its stable and open nature, especially when compared to Microsoft's counterpart: Windows Server.

The second argument critics have is the monetization of open-source projects, namely how can people make livings giving away knowledge? Many companies have adapted to support the open-source community while still making a lot of money. Microsoft, Google, and Facebook are among the biggest contributors to open-source projects (including Linux, Ubuntu, React, and TensorFlow) and are some of the most profitable companies in the world. So how do these projects support themselves? The easiest ways are usually advertisements, donations, and/or licenses (\cite{OpenSouceMakeMoney}). A great example is a project called ChibiOS: an operating system for microcontrollers. While open-source, the operating system is licensed for free for hobbyists, educators, and projects that use the GPL3 license while requiring a contract with companies who try to commercialize their product that utilizes ChibiOS (\cite{ChibiOS}). Another great way for open-source projects and companies to be self-sufficient is to follow a service model instead of a product model. The main product sold could be support service rather than the product itself. Canonical - makers and maintainers of the most popular Linux distribution Ubuntu - follow this exact model. Instead of selling the operating system to their customers, they offer it for free and make it open source, and sell support for it instead. Big businesses will usually purchase this support so they don't have to worry about fixing issues they are unfamiliar with, while others can get a top of the line operating system for free! Since it is open source, customers can submit their own changes to make the operating system better for everyone. 

\subsubsection{Universal Basic Income}

