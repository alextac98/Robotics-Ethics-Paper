\section{Mid-Term (20 - 100 years)}
\label{sec:mid}
This in-between period will prove to be a defining period for humanity. If every time period this paper talked about had a theme, this one's would be %TODO: Come up with a theme.
With an ever-increasing globalization for both technology and culture, we will either see an extreme separation of wealth or a great equalizer where nearly everyone has equal opportunity to compete. Experts generally agree that such a difference will shift power to just a few individuals, both politically and socially in a way that can't be balanced or checked (\cite{WhatsWrongWithWealthDistribution}).
\subsection{Outlook on Robotics and Jobs}

\subsection{Wealth Distribution Solutions}

So how can we as a global society equalize economic and social opportunities as best as possible? Author Matshona Dhliwayo once said "Knowledge is wealth, wisdom is treasure, understanding is riches, and ignorance is poverty." % Add citation
Therefore, in order to avoid a power gap, we should systematically implement wealth distribution solutions. Some keen-eyed readers may say this sounds like a socialistic society, but there is one major difference: information vs material. Socialism, as defined by Busky, is characterized by the shared ownership of production and enterprises (\cite{WhatIsSocialism}). While many interpretations have existed, many radical implementations of socialism have largely failed, such as most communist regimes. These implementations have focused on shared ownership of goods and organizations through government mandates. A relatively free and competitive market, however, encourages competition and innovation, which pushes humanity forward. While some government regulation may help redistribute traditional wealth through universal basic incomes, the most powerful way of paving a rich path for humanity is by encouraging the free and open sharing of information and technology.

\subsubsection{Open-Source Movement} \label{subsec:open-source}
A potentially different mentality should be used for our future: sharing information rather than ownership. This encourages innovation since inventors can use existing technology to develop new concepts, and forces competition through the requirement of being the best in order to continue. The sharing of information also reduces the likelihood of powerful technology leaders from controlling society. W

\subsubsection{Universal Basic Income}

